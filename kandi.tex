\documentclass[finnish]{tktltiki2}
% vim: set textwidth=100 :
% rubber: module pdftex

\usepackage[utf8]{inputenc}
\usepackage[T1]{fontenc}
\usepackage{lmodern}
\usepackage{microtype}
\usepackage{amsfonts,amsmath,amssymb,amsthm,booktabs,color,enumitem,graphicx}
\usepackage[pdftex,hidelinks]{hyperref}

\usepackage{listings}
\renewcommand{\lstlistingname}{Listaus}
\lstset{basicstyle=\ttfamily,captionpos=b,frame=single,language=Haskell,literate={ä}{{\"a}}1}

\makeatletter
\AtBeginDocument{\hypersetup{pdftitle = {\@title}, pdfauthor = {\@author}}}
\makeatother

\usepackage[fixlanguage]{babelbib}
\selectbiblanguage{finnish}

\usepackage[nottoc]{tocbibind}
\settocbibname{Lähteet}

\title{Haskell ja monadinen imperatiivisuus}
\author{Tuomas Starck}
\date{\today}
\level{Kandidaatintutkielma}

\abstract{ \\
Haskell on puhdas, laiska ja funktionaalinen ohjelmointikieli, joka tarjoaa vaihtoehtoisen
ohjelmointiparadigman valtavirran imperatiivisille ja oliopohjaisille kielille. Siirräntä
tarkoittaa tietokoneohjelman vuorovaikutusta ympäristönsä kanssa.

Tutkielma esittelee puhtauteen ja siirräntään liittyviä ongelmia ja tarkastelee kuinka Haskell
mahdollistaa siirrännän ohjelmoinnin hyödyntäen kategoriateorista lainattua monadien konseptia.
Teoreettisen viitekehyksen lisäksi tutkielma selittää kuinka Haskellin IO-monadi toimii ja kuinka
sen avulla on helppoa toteuttaa siirräntää.
}

\classification{\protext{ \\
Software and its engineering $\rightarrow$ Software notations and tools $\rightarrow$ General
programming languages $\rightarrow$ Language types $\rightarrow$ Functional languages}}

\keywords{Haskell, siirräntä, funktionaalinen ohjelmointi, monadit}

\begin{document}

\frontmatter

\maketitle
\makeabstract

\tableofcontents

\mainmatter

\newcommand{\arr}[0]{\rightarrow}
\newcommand{\code}[1]{\texttt{#1}}
\newcommand{\fixme}[1]{\emph{#1}}


\section{Johdanto}

% Jäsennelty asianmukaisesti (kenelle, miksi, millaisessa ympäristössä; ratkaisun lähestymistapa;
%  tutkimuskysymys, tulokset ja impakti; loppulukujen roolitus), pituus 1,5 - 2 s.
% Miksi funktionaalisuus: John Hughes, Why Functional Programming Matters{why90}

Hyvälle tietokoneohjelmalle on monia vaatimuksia kuten esimerkiksi ohjelman oikeellisuus,
soveltuvuus käyttötapaukseensa, tehokkuus ja mahdollisuus ymmärtää ja laajentaa ohjelmaa.
Vastaavasti hyvän ohjelmointikielen tulisi tukea ohjelmoijan päämääärää hyvän ohjelman luomisessa.
Tietojenkäsittelyn historian aikana on kehitetty useita ohjelmointikieliä, joista nykyisin
yleisimmin käytössä olevat hyödyntävät imperatiivista ohjelmointiparadigmaa. Tämä tutkielma
tarkastelee ohjelmointikieltä \emph{Haskell}, joka ei ole imperatiivinen kieli vaan kuuluu
funktionaalisten kielien joukkoon.

Imperatiivisen paradigman ohjelma mallintuu pelkistetyssä tarkastelussa suoraan matalan tason
konekäskyiksi, joten siinä mielessä se on johdonmukainen tapa lähestyä ohjelmointia. Laskenta
mikroprosessorin avulla tapahtuu siten, että prosessorin ja muistin tila muuttuu konekäskyjä
suoritettaessa ja vastaavasti imperatiivisen ohjelman käskyjen suoritus muuttaa ohjelman tilaa.
Funktionaalisen paradigman kielet sen sijaan muodostuvat lausekkeista ja laskenta tarkoittaa
kyseisten lausekkeiden sieventämistä.

Siirräntä (engl. \emph{input/output}) tarkoittaa tietokoneohjelman vuorovaikutusta ympäristönsä
kanssa. Haskell on paitsi funktionaalinen niin myös puhdas ja laiska kieli. Puhtauden määritelmään
kuuluu että ohjelma ei saa vuorovaikuttaa ympäristönsä kanssa eli ohjelmalla ei tule olla
sivuvaikutuksia. Koska sivuvaikutukset määritelmällisesti karkaavat ohjelman semantiikan
ulkopuolelle, ne saattavat sekä tehdä ohjelman ymmärtämisen vaikeaksi että vaikeuttavat ohjelman
oikeellisuuden päättelyä.

Toisaalta tietokoneohjelma on hyödyllinen vain mikäli se voi ottaa vastaan syötettä ja ilmaista
laskennan tuloksen jollain tavalla. Toisin sanoen siirräntä on välttämätön ominaisuus hyvälle
ohjelmalle eikä mikään vakavasti otettava ohjelmointikieli voi olla huomioimatta tätä tarvetta. Myös
Haskellin oli ratkaistava tämä ilmeinen ristiriita.

Tämä tutkielma perehtyy Haskellin valintaan lainata kategoriateorian monadia teoreettiseksi malliksi
sille, kuinka siirräntä on eristetty kielen semantiikan puitteissa omaan kontekstiinsa ja
funktionaalisuuden toivotut ominaisuudet on säilytetty käyttökelpoisina. Tutkielma tarkastelee myös
kuinka siirrännän totettaminen on mahdollistettu ja tehty helpoksi Haskellin kielen ominaisuuksien
avulla. Mutta aluksi tutkielma luo lyhyen katsauksen Haskellin syntyyn, jotta tekniset ongelmat
saavat myös historiallisen kontekstin.

% luonteeltaan arvaamattomia ja ohjelman toiminnasta on helpompi vetää johtopäätöksiä ilman
% treats programs as evaluating mathematical functions and avoids state and mutable data


\section{Haskell} %% {hoh07}

\subsection{Historia}

Haskell sai alkunsa vuonna 1987 Functional Programming and Computer Architecture -konferenssin
yhteydessä pidetyssä tapaamisessa, jossa päätettiin suunnitella ja kehittää uusi ohjelmointikieli.
Ennen Haskellia oli olemassa lukuisia funktionaalisia ohjelmointikieliä, mutta vaihtoehtojen
ylitarjonta sirpaloi funktionaalisen ohjelmoinnin kentän ja hidasti sen edistystä. Kielen
kehittämistä varten perustettiin komitea, jonka tavoitteena oli luoda yksi yhtenäinen
funktionaalinen ohjelmointikieli. Komitean yksi ensimmäisistä saavutuksista oli uuden kielen
nimeäminen Haskelliksi.

\subsection{Laiskuus}
\label{subsec:laiskuus}

Haskell-komitean merkittävin yhteinen tavoite oli tehdä Haskellista laiska. Laiskuudella
tarkoitetaan sitä, että ohjelman lausekkeet evaluoidaan vasta silloin, kun niiden palauttama arvo on
tarpeen. Laiska suoritus mahdollistaa muun muassa äärettömien tietorakenteiden määrittelyn ja
käsittelyn, sillä Haskellin suoritusympäristö ei pyri evaluoimaan niitä loppuun saakka, joten
ohjelma ei jää jumiin.

Yleisesti ottaen Haskell-komitean pyrkimys oli käyttää olemassa olevia ja hyvin tunnettuja
tekniikoita suunnittelutyönsä pohjana. Laiskuudesta oli noin vuosikymmenen verran kokemusta
Haskellin synnyn aikaan, joten se ei ollut uusi keksintö.

Laiskuus ei ole täysin ongelmaton valinta. Laiskan suorituksen on pidettävä kirjaa siitä, mitkä osat
ohjelmaa ovat suorittamatta, ja se kasvattaa yleiskustannuksia. Kirjanpidosta seuraava hidastus on
kuitenkin vakioinen ja verrattain pieni. Suurempi ongelma on se, että laiskan suorituksen
tilavaativuutta on vaikea arvioida ja se saattaa kasvaa ylilineaarisesti. Haskelliin on siksi
lisätty sekä tietorakenteita, jotka eivät ole laiskoja, että keino pakottaa lausekkeen evaluaatio,
jolloin laiskuuden voi poistaa sellaisesta kohdasta ohjelmaa, jossa se on haitaksi.

\subsection{Puhtaus}

% FIXME siirräntä määritelty kahdesti?
%       lisää esimerkkejä ehkä?

% syöte, tuloste
% jaetun muistin luku, kirjoitus
% verkko...

Laiskuuden välitön seuraus on se, että evaluointijärjestys määräytyy tarpeen mukaan, jolloin
vastaavasti sivuvaikutusten käsittelyä on vaikeaa toteuttaa mielekkäästi. Sivuvaikutuksilla
tarkoitetaan esimerkiksi siirrännän käsittelyä tai yleisesti ottaen mitä tahansa ulkoista
tilanmuutosta.

Haskell-komitean ratkaisu oli tehdä Haskellista puhdas kieli, jolloin funktioilla ei ole lainkaan
sivuvaikutuksia. Tässä mielessä Haskellin funktiot ovat paitsi nimeltään niin myös merkitykseltään
vastaavanlaisia matemaattisten funktioiden kanssa -- molemmat palauttavat samalla syötteellä aina
saman arvon.

Puhtauden vaikutukset ovat kaikenkattavat. Koska sivuvaikutuksia ei sallittu, oli Haskellin
alkuaikoina siirräntä tuskallisen hankalaa. Pakon edessä kehitettiin monadinen siirräntä, jota
voidaan pitää yhtenä Haskellin merkittävimmistä panostuksista ohjelmointikielten kehitykselle.
Keskustelu siitä, onko puhtaan kielen ja monadisten sivuvaikutusten yhdistelmä paras lähestymistapa
ohjelmistojen kehitykselle, on yhä kesken, mutta Haskellin tarjoama ehdotus on ainakin mullistava ja
elegantti. Lisäksi se on synnyttänyt paljon tutkimusta monadeista ja tilan kapseloinnista.

\subsection{Tyyppiluokat}

Tyyppiluokkia pidetään nykyisin yhtenä Haskellin tunnusmerkillisimmistä ominaisuuksista. Alunperin
tyyppiluokat tuotiin osaksi kieltä, koska oli tarve ratkaista yhtäsuuruuden vertaamisen ja
numeeristen operaatioiden ylikuormitus. Haskell-komitean tavoite oli suunnitella kieli käyttäen
olemassa olevia ja perusteellisesti tunnettuja menetelmiä, mutta Haskellia edeltäneissä
funktionaalisissa kielissä, kuten Mirandassa tai Standard ML:ssä, ei ylikuormitusta oltu ratkaistu
yhtenäisellä tavalla. Lisäksi aiemmin käytetyissä menetelmissä oli heikkouksia, kuten esimerkiksi
Mirandan polymorfinen yhtäsuuruusvertailu~\fixme{laajenna?}, joka saattoi johtaa ajonaikaisiin
virhetilanteisiin.

Tyyppiluokat mahdollistavat nimensä mukaisesti tyyppien luokittelun sen perusteella, mitä
ylikuormitettuja operaatioita ne tukevat. Esimerkiksi Haskellissa luokkaan \code{Eq} kuuluvalle
tyypille on toteutettu yhtäsuuruusvertailu. Tällöin käännösvaiheessa on mahdollista tarkistaa, että
\code{==}-operaattoria käytetään vain sellaisten tyyppien kesken, jotka kuuluvat
\code{Eq}-luokkaan. Vaikka tyyppiluokat olivat uusi keksintö eikä niiden toimivuutta oltu
todistettu aiemmin, Haskell-komitea päätyi omaksumaan ne osaksi kieltä, sillä niiden avulla kaikkien
tyyppien ylikuormitus oli ratkaistavissa yhtenäisellä, järjestelmällisellä ja modulaarisella
tavalla.

Jälkikäteen ajateltuna tyyppiluokkien ottaminen osaksi Haskellia oli onnekas valinta,
jolla on ollut kauaskantoisia vaikutuksia, sillä tyyppiluokat ovat monikäyttöinen työkalu ja niiden
avulla voi tehdä paljon muutakin kuin toteuttaa ylikuormituksen. Yksi merkittävä edistysaskel
Haskellin ja tyyppiluokkien kehitykselle oli ehdotus tyyppimuodostajan parametroimisesta
\fixme{Onko parametroimisesta sana?}
tyyppimuuttujan sijaan. Tämä mahdollisti korkeamman asteen polymorfismin (engl. \emph{higher-kinded
polymorphism}), jonka välitön sovellus oli monadien toteuttaminen.

\subsection{Syntaksi}

%% FIXME Onko tää yhtään enää oikein muun tekstin kontekstissa!?

Koodiesimerkkejä käytetään ongelmien ja ratkaisujen havainnollistamiseen. Esimerkit seuraavat
syntaksiltaan Haskellia silloinkin, kun ne ovat pseudokoodia, joka ei käänny Haskell-ohjelmaksi.
Listauksessa~\ref{lst:esittely} esitellään käytetty syntaksi, jossa on kaksi osaa: funktion tyypin
määrittely ja varsinainen funktion määrittely. Tyyppimäärittely alkaa funktion nimellä, jonka
jälkeen tulee kaksi kaksoispistettä, jota seuraa mahdolliset parametrit ja paluuarvo eroteltuna
nuolella (\code{->}). Funktion määrittely alkaa myös funktion nimellä, jota seuraa mahdolliset
parametrit (tässä esimerkissä \code{x} ja \code{y}), joiden jälkeen tulee yhtäsuuruusmerkki ja
funktion vartalo.

\begin{lstlisting}[float,label={lst:esittely},caption={Syntaksin esittely}]
summa :: Int -> Int -> Int
summa x y = x + y
\end{lstlisting}


\section{Siirrännän vaikeus}

% Ollaanko me määritelty siirräntä missään vaiheessa? Tulostus, syötteen lukeminen, FFI,
% tilamanipulaatiot...

% Purkaa auki teknisemmälle yleisölle samalla käsitteitä määritellen tutkimuskysymyksen, ratkaisuille
%  ympäristöstä esiintulevat vaatimukset ja keskeiset käsitteet näiden asioiden esittelemiseksi.
% Antaa riittävästi termistöä ja kysymyksenasetteluja, jotta myöhempänä teksissä on mahdollisuus
%  analyysiin ja evaluointiin.

Haskellin kehityksen alkutaipaleella pohdittiin siirrännän toteuttamista ja siihen liittyviä
ongelmia. Valmista hyvänä pidettyä ratkaisua ei ollut olemassa, sillä olemassa olevien puhtaiden
funktionaalisten kielten siirräntää pidettiin epätyydyttävänä tai kielessä oli sallittu
sivuvaikutukset.

\subsection{Puhtaus}

Puhtaasta laskennasta on monia hyötyjä. \fixme{Parempi esimerkki!} Esimerkiksi rinnakkaista
ohjelmointia pidetään perinteisesti vaikeana, ja yksi merkittävä syy ohjelmointivirheille on se,
että suorituksen eri haarat lukevat tai muuttavat jaettua tilaa virheellisesti. Puhdas rinnakkainen
laskenta sen sijaan on sivuvaikutuksetonta, joten se on immuuni samoille ongelmille.

% https://en.wikipedia.org/wiki/Dijkstra_Prize

Sivuvaikutusten puute tarkoittaa myös sitä, että Haskell on viitteellisesti läpinäkyvä (engl.
\emph{referentially transparent}). Toisin sanoen minkä tahansa Haskellin lausekkeen voi korvata
kyseisen lausekkeen arvolla ilman, että ohjelman merkitys muuttuu. Käytännössä viitteellinen
läpinäkyvyys mahdollistaa optimoinnin, jossa kääntäjä pelkistää kaikki toistuvat viitteet yhdeksi.
Esimerkiksi listauksessa~\ref{lst:kuutio} muuttujan \code{x} arvo täytyy selvittää vain kerran,
jonka jälkeen saadulla \code{x}:n arvolla korvataan jokainen \code{x}:n ilmentymä ja kertolaskut
voidaan laskea.

\begin{lstlisting}[float,label={lst:kuutio},caption={Puhdasta laskentaa}]
kuutio x = x * x * x
\end{lstlisting}

Esimerkin yksinkertaisuus saattaa hämätä, joten on syytä muistaa, että \code{x} voi olla
laskennallisesti mielivaltaisen haastava funktio, joten optimointi voi olla merkittävä. Lisäksi
ilman puhtautta \code{x}:n arvo saattaisi olla riippuvainen esimerkiksi ulkoisesta syötteestä,
jolloin optimointia ei voisi tehdä yhtä huolettomasti. Toisaalta aina puhtaudesta johtuva
optimointi ei johda toivottuun lopputulokseen.

\begin{lstlisting}[float,label={lst:samat},caption={Kaksi samaa funktiota}]
lueKaksiRivia = { readLine; readLine }
\end{lstlisting}

Esimerkissä~\ref{lst:samat} käytetään kahdesti samaa funktiota, jonka tarkoitus on lukea käyttäjän
antama syöte. Koska kääntäjä luottaa puhtauteen ja optimoi identtiset funktiot yhdeksi
evaluaatioksi, tapahtuu varsinainen luku vain kerran, sillä sen jälkeen molempien funktioiden
ilmentymä voidaan korvata kerran tehdyn evaluaation palauttamalla arvolla. Siirrännän käsittelyssä
puhtaus ja sen seuraukset eivät aina ole toivottavia.

\subsection{Laiskuus}

Laiska suoritus sopii hyvin yhteen puhtaan funktionaalisen kielen kanssa, mutta sekin tekee
siirrännän käsittelyn vaikeaksi. Listauksessa~\ref{lst:tulostus} on esimerkkiohjelma, jossa
määritellään funktio \code{sanoHaskell}, jonka tarkoitus on tulostaa merkkijono `Haskell'
oletustulostusvirtaan.

\begin{lstlisting}[float,label={lst:tulostus},caption={Tulostava esimerkkiohjelma}]
sanoHaskell = print "Haskell"
\end{lstlisting}

Imperatiiviseen paradigmaan tottunut saattaisi olettaa, että ohjelma tulostaa `Haskell' ja lopettaa.
Laiska suoritus kuitenkin johtaa siihen, ettei mitään tulostu. Funktiota \code{print} ei koskaan
evaluoida, sillä sivuvaikutuksista ei välitetä eikä funktio palauta mitään sellaista arvoa, joka on
tarpeen ohjelman suorittamiseksi. Toisin sanoen ohjelman tulostus optimoidaan pois, vaikka se on
ohjelman ainoa varsinainen tarkoitus.

Laiskuuteen ja siirräntään liittyy myös toinen ongelma, joka on hieman hienovaraisempi.
Listauksen~\ref{lst:jarjestys} esimerkissä määritellään funktio \code{sanoHeiMaailma}, jonka on
tarkoitus suorittaa kaksi merkkijonon tulostusta. Imperatiivisissa ohjelmissa suoritus on
tavanomaisesti ahnetta ja ohjelman suoritus seuraa kirjoitetun koodin rakennetta, mutta sama sääntö
ei päde laiskaan evaluointiin. Tästä seuraa se, ettei ole olemassa takeita siitä, missä
järjestyksessä esimerkkiohjelman tulostukset suoritetaan.

\begin{lstlisting}[float,label={lst:jarjestys},caption={Kahden tulostuksen pseudokoodi}]
sanoHeiMaailma = { print "Hei"; print " maailma" }
\end{lstlisting}

Toisin sanoen siinäkin tapauksessa, että tulostukset suoritettaisiin laiskuudesta huolimatta,
saattaa merkkijono `maailma' tulostua ennen merkkijonoa `Hei'. Tämä on luonnollisesti merkittävä
ongelma, sillä yleisessä käyttötapauksessa siirrännän käsittelyn tulee tapahtua määrätyssä
järjestyksessä, jotta ohjelma toimii tarkoituksenmukaisesti.

\subsection{Kohti imperatiivisuutta}

Listauksessa~\ref{lst:ratkaisu} on ohjelma, jonka käyttötarkoitus on sama kuin aiemmassa
listauksessa~\ref{lst:samat} eli ohjelman tulisi lukea syöte kahdesti, mutta toteutusta on
laajennettu siten, että puhtauden ja laiskuuden aiheuttamat ongelmat on pyritty ratkaisemaan.
Lisäksi tällä kertaa käytettyjen funktioiden tyyppimäärittelyt ovat kirjoitettu näkyviin.
Ratkaisussa on käytetty tekaistuja parametreja, joiden avulla suoritusjärjestys pakotetaan
halutunlaiseksi. Tekaistujen parametrien tyyppi ei ole tärkeä, ja listauksessa käytetty \code{Int}
on valittu vain yhtenä esimerkkinä kaikista mahdollisuuksista.

\begin{lstlisting}[float,label={lst:ratkaisu},caption={Suoritusjärjestys riippuvuuksien avulla}]
lueRivi       :: Int -> ( String,  Int)
lueKaksiRivia :: Int -> ([String], Int)

lueKaksiRivia p0 = ([s1, s2], p2) where
  (s1, p1) = lueRivi p0
  (s2, p2) = lueRivi p1
\end{lstlisting}

Funktio \code{lueKaksiRivia} palauttaa arvot \code{s1} ja \code{s2}, jotka edustavat niitä
merkkijonosyötteitä, joiden lukeminen on ohjelman varsinainen tavoite. Lisäksi ohjelma palauttaa
tekaistun arvon \code{p2}. Ohjelman toiminta on helpoin päätellä seuraamalla ohjelman suoritusta
käänteisessä järjestyksessä. Jotta \code{p2}:n arvo saadaan selville, on välttämätöntä suorittaa
funktio \code{lueRivi p1}. Jotta se voidaan suorittaa, täytyy saada selville arvo \code{p1}, joka
vuorostaan vaatii funktion \code{lueRivi p0} suorituksen ensin. Tällöin tekaistuilla parametreilla
on saavutettu se, että funktioilla \code{lueRivi} on muuttujien arvoon liittyvä riippuvuus, jolloin
laiska suoritus ei voi poimia suoritusjärjestystä mielivaltaisesti.

Samalla vältetään puhtaudesta juontuvat ongelmat. Funktioita \code{lueRivi p0} ja \code{lueRivi p1}
ei voi pelkistää yhdeksi evaluaatioksi, sillä ne saavat eri parametrit.

Funktion \code{lueKaksiRivia} saama parametri \code{p0} ei ole välttämätön tämän esimerkin kannalta,
mutta on tärkeä laajemmassa kontekstissa, sillä se mahdollistaa funktioiden koostamisen (engl.
\emph{function composition}). Jos funktiolle \code{lueRivi} ei voisi antaa mitään parametria, ei
puhtauden ja laiskuuden ongelmia olisi voitu ratkaista tekaistuilla parametreilla. Vastaavasti ilman
parametria \code{p0} olisi funktiolla \code{lueKaksiRivia} sama ongelma laajemman ohjelman
kontekstissa. Tästä syystä ei ole lainkaan sattumaa, että käytettyjen funktioiden tyyppimääritelmät
ovat hyvin samankaltaiset, sillä molempien funktioiden on tarkoitus toteuttaa siirräntää puhtaassa
ja laiskassa kielessä.

Vaikka tämän esimerkin ohjelman voi nähdä toimivan tarkoituksenmukaisella tavalla, on se samalla
muuttunut huomattavasti monimutkaisemmaksi. Ohjelma on vaikeaselkoinen, koska eri parametrien
tarkoitus ei ole ilmeinen, ja toisaalta tekaistujen parametrien käyttö ohjelman suorituksen
ohjaamiseen on semanttisesti ongelmallista.

\subsection{Tavoitteet}

Siirrännän toteuttaminen puhtaassa ja laiskassa kielessä ei ole itsestäänselvä asia, jonka vuoksi
jotkut funktionaaliset ohjelmointikielet ovat luopuneet puhtauden vaatimuksesta ja hyväksyneet
sivuvaikutukset osaksi siirrännän toteutusta. Koska Haskellin suunnittelun kulmakivi alusta alkaen
on ollut puhtaus ja laiskuus, on ratkaisun siirrännän toteuttamiseksi oltava yhteensopiva niiden
kanssa.

Ohjelmat yleensä kokonaisuuksia, jotka syntyvät pienemmistä osatekijöistä. Funktionaalisten kielien
luonnollinen vahvuus on perinteisesti ollut se, että funktioita on helppo koostaa osaksi korkeamman
asteen funktioita (engl. \emph{higher-order function}). Haskellin siirrännän ratkaisun olisi tulisi
säilyttää tämä ominaisuus, jotta kielen käytettävyys tai ilmaisuvoima ei heikkene, jos ohjelman
tarvitsee tehdä siirräntää.

Lisäksi ratkaisun tulisi olla soveltuva ohjelmointikieleen yleisesti. Suotavia ominaisuuksia ovat
mm. ratkaisun yleishyödyllisyys ja laajennettavuus, tehokkuus ja toteutuksen helppous. Toisin sanoen
olisi kätevää, jos siirrännän haasteet olisivat ratkaistavissa siten, että samaa mallia voisi
soveltaa muihin Haskellin kipukohtiin. Ja niin kuin kaikkien ohjelmointiin liittyvien ongelmien
kanssa ei ratkaisu voi olla laskennallisesti liian vaativa tai sen käyttökelpoisuus heikkenee.


\section{Kategoriateoria}

\subsection{Historia}

Alunperin kategoriateoria ja monadit tulivat osaksi tietojenkäsittelytiedettä vuonna 1989 Eugenio
Moggin julkaisun "Computational Lambda-calculus and Monads"~\cite{clcam89,nocam91} myötä. Moggi
tutki ohjelmointikieliä ja keinoja, joilla tietokoneohjelmista voidaan vetää loogisia
johtopäätöksiä, kuten esimerkiksi osoittaa kahden ohjelman yhdenmukaisuus.

Lambdakalkyyli oli aiemmin tarjonnut matemaattisen teoreettisen pohjan ohjelmien yhdenmukaisuuden
vertailuun. Lambdakalkyylin ei kuitenkaan ollut ongelmaton valinta, sillä se vaati ohjelman tai
proseduurin merkityksen pelkistämisen täydelliseksi funktioksi (engl. \emph{total function}).
Tällöin menetetään mahdollisuus tarkastella muun muassa pysähtyvyyttä, epädeterministisyyttä ja
sivuvaikutuksia, jotka ovat yleisiä oikeissa ohjelmissa.

% (1) https://en.wikipedia.org/wiki/Denotational_semantics
%
% (2) "Moggi seeks to capture the intuitive notion of a computation in the most general setting
% possible." -- Kategoriateoreettinen lähestymistapa tarjoaa yhdenmukaisuudelle ja pysähtymiselle
% luonnollisen selityksen. -- "That existence and equivalence are so readily definable suggests that
% far from just being a way to hide I/O or statefulness, Monads and their attendant laws express the
% conditions satisfied by a computation strategy."

Moggin valinta oli käyttää kategoriateorian tarjoamaa semanttista kehystä laskennan mallintamiseen.
Erityisesti Moggin innovaatio oli hyödyntää monadeja, joka mahdollisti ohjelmointikielen
ominaisuuksien jakamisen erillisiin osiin niiden semantiikan perusteella. Tällöin esimerkiksi
siirräntä, poikkeukset tai muut toisistaan poikkeavat ominaisuudet saivat oman erillisen kontekstin,
jossa niitä voitiin tarkastella.

Philip Wadler havaitsi, että Moggin käyttämä strategia rakenteen luomiseen semantiikan avulla
soveltuisi myös funktionaalisten ohjelmien jäsentämiseen. Toisin sanoen siinä missä Moggi käytti
monadeja ohjelmien kuvaamiseen, päätti Wadler hyödyntää monadeja ohjelmien
ilmaisemiseen~\cite{cm90,teofp92}.

\subsection{Kategoria}

Kategoriateoria on abstrakti matemaattinen työkalu, joka mahdollistaa monien matemaattisten
käsitteiden formalisoinnin. Kategoriateoreettinen lähestymistapa mahdollistaa erilaisten
matemaattisten rakenteiden (kuten esimerkiksi joukkojen, ryhmien tai vektoriavaruuksien) tarkastelun
rakenteen säilyttävällä tavalla, jolloin kategorian tutkiminen sallii yhteisten piirteiden
kuvaamisen ja todistamisen.

Kategoria muodostuu kahdesta tekijästä, jotka ovat merkityksellisiä Haskellin näkökulmasta: olioista
ja morfismeista. Morfismeja (engl. \emph{morphisms}) kutsutaan myös nuoliksi (engl. \emph{arrow}),
mutta nuolella on Haskellissa eri merkitys, joten jatkossa pitäydytään nimityksessä morfismi.
Merkintä $f : A \rightarrow B$ tarkoittaa morfismia $f$ lähdeoliolta $A$ kohdeoliolle $B$ jossain
oletetussa kategoriassa.

Yhdiste (engl. \emph{composition}) on binäärioperaatio morfismeille siten, että jos samassa
kategoriassa on olemassa morfismit $f : A \arr B$ ja $g : B \arr C$, on oltava myös niiden
yhdistemorfismi $g \circ f : A \arr C$. Lisäksi morfismit noudattavat kahta sääntöä. Ensinnäkin
morfismien yhdisteen tulee olla assosiatiivinen. Toisin sanoen:

$$f \circ (g \circ h) = (f \circ g) \circ h$$

Toiseksi missä tahansa kategoriassa $\mathcal{C}$ on olemassa jokaiselle oliolle $A$
identiteettimorfismi $1_A : A \arr A$. Tarkemmin sanottuna jokaiselle morfismille $f : A \arr B$
pätee:

$$ 1_A \circ f = f = f \circ 1_B $$

%% https://wiki.haskell.org/Hask

Miten sitten kategoriat liittyvät Haskelliin? Haskellin yhteydessä puhutaan usein kategoriasta
nimeltään \code{Hask}, joka tarjoaa teoreettisen kehyksen ohjelmointikielen itsensä tarkasteluun.
\code{Hask}-kategorian kontekstissa Haskellin tyypit vastaavat kategoriateorian olioita ja funktiot
vastaavat morfismeja. Lisäksi funktiokompositio \verb|(.)| toimii Haskellissa yhdistemorfismina,
sillä se on assosiatiivinen, ja funktio \code{id} toteuttaa identiteettimorfismin vaatimuksen.

Päällisin puolin näyttää siltä, että Haskellin ominaisuudet luovat suoraan kategorian siten kuin
kategoriateoria sen määrittelee. Tämä ei kuitenkaan pidä paikkaansa, sillä on olemassa rajatapauksia
erityisesti määrittelemättömien ja tyhjien arvojen kanssa, joissa Haskell ei käyttäydy
kategoriateorian sääntöjen mukaisesti. Sen sijaan ei ole väärin ajatella, että Haskell on omaksunut
vaikutteita kategoriateoriasta ja sen seurauksena Haskellin monadit ovat analogisia kategoriateorian
monadien kanssa.

% The category must satisfy an identity axiom and an associative axiom
% which is analogous to the monoid axioms.

% \subsection{Monoidi}
% \fixme{Onko tarpeen puhua monoideista?}

% \subsection{Funktori}
% \fixme{Onko tarpeen puhua funktoreista?}

\subsection{Monadi}

\begin{lstlisting}[float,float,label={lst:monad},caption={Monad-tyyppiluokka}]
class Applicative m => Monad m where
  (>>=)  :: forall a b. m a -> (a -> m b) -> m b
  (>>)   :: forall a b. m a -> m b -> m b
  return :: a -> m a
\end{lstlisting}

Monadi on yksi kategoriateorian yleisimmin hyödynnetyistä rakenteista, joka on osoittautunut
monikäyttöiseksi työkaluksi myös ohjelmointikieliä tarkasteltaessa. Erityisesti funktionaalisten
kielien tapauksessa monadit sallivat laskennan kuvaamisen riittävän abstraktilla tavalla, jotta ne
ovat päätyneet ratkaisemaan lukuisia käyttötapauksia. Pelkästään Haskellin standardikirjasto tarjoaa
monadeihin tukeutuvia työkaluja paitsi siirrännän niin myös tilan, virheenkäsittelyn ja
epädeterministisen laskennan tarpeisiin.

Monadit ovat toteutettu Haskellissa tyyppiluokkana. Listauksessa~\ref{lst:monad} on lainaus
Haskell-kääntäjän GHC (\emph{Glasgow Haskell Compiler}) lähdekoodista, josta on nähtävissä
\code{Monad}-luokka ja tyyppimääritelmät sen kolmelle metodille (eli luokan funktiolle). Yksittäiset
tyypit, joille periytetään monadin ominaisuudet, määrittelevät tarkoitukseensa sopivan toteutuksen
vaadituille metodeille.

Jotta luotu tyyppi käyttäytyy oikean monadin tavoin, täytyy metodien toteutuksen seurata kolmea
sääntöä, joita kutsutaan myös monadilaeiksi (engl. \emph{monad laws}). Säännöt pitävät huolta siitä,
että monadien yhdisteet toimivat assosiatiivisesti ja että metodit säilyttävät toimintojen samuuden.
Haskellin ilmaisuvoima ei riitä siihen, että kääntäjä pystyisi todistamaan jonkun tyypin täyttävän
monadilait, joten se tehtävä jää ohjelmoijan vastuulle, jos hän tahtoo luoda uuden monadin.

\subsection{Monadin toteutus}

Mitä monadiin kuuluvat funktiot sitten tekevät? Ensimmäinen niistä on $(>>=)$, joka luetaan
\emph{bind} tai suomeksi \emph{sidos}. Kuten nimi viittaa sidoksen avulla voi yhdistää kaksi monadin
toiminto. Tyypillisesti sidosta käytetään sisämerkintänä argumenttiensa välissä. Esimerkiksi $a >>=
f$ tarkoittaa, että toimisto \code{a} ja funktio \code{f} ketjutetaan yhteen siten, että evaluaation
aikaan toiminto \code{a} suoritetaan, sen paluuarvo sovelletaan funktioon \code{f}, josta saadaan
koko lauseen paluuarvo.

Toinen monadin funktio $(>>)$ luetaan \emph{then}, jonka voi kääntää suomeksi \emph{sitten}. Se
toimii muutoin samankaltaisesti sidoksen kanssa, mutta se ei huomioi ensimmäisen toiminnon
paluuarvoa. Toisin sanoen sitä voi käyttää yhdistämään toimintoja, joiden paluuarvoilla ei ole
merkitystä. Jos tarkkoja ollaan $(>>)$ ei ole teoriassa pakollinen toimivan monadin toteuttamiseksi,
mutta käytännön ohjelmoinnin ja koodin selkeyden näkökulmasta se on hyödyllinen.

Useissa imperatiivisissa ohjelmointikielissä tunnetaan avainsana \code{return}, jonka tarkoitus on
keskeyttää aliohjelman suoritus ja palauttaa jokin arvo. Haskellissa \code{return} on funktio, joka
nostaa argumenttinsa monadin kontekstiin. Toisin sanoen \code{return a} evaluoituu monadiseksi
tyypiksi, joka pitää sisällään arvon \code{a}. Funktion \code{return} nimi on valitettava sillä se
aiheuttaa paljon hämmennystä uusien Haskellin harrastajien keskuudessa.

\subsection{Toimiva ohjelma}

Listauksessa~\ref{lst:then} on päivitetty versio luvun~\ref{subsec:laiskuus} esimerkistä. Ohjelmaa
on muutettu siten, että tulostustoiminnot ovat yhdistetty \emph{then}-funktiolla, jonka lisäksi
tyyppimääritelmä on kirjoitettu näkyviin sen korostamiseksi, että toiminnot ovat suoritettava
IO-monadin kontekstissa. Tällä kertaa ohjelma on täysin toimiva ja se tulostaa merkkijonot
toivotulla tavalla.

\begin{lstlisting}[float,label={lst:then},caption={Toimiva tulostus}]
sanoHeiMaailma :: IO ()
sanoHeiMaailma = putStr "Hei" >> putStr " maailma"
\end{lstlisting}

Päällepäin katsottuna ei kuitenkaan ole ilmeistä, kuinka ohjelma eroaa
listauksesta~\ref{lst:jarjestys}? Näemme että ohjelma hyödyntää \code{then}-funktiota, mutta mikä
mekanismi pitää huolta, että tulostukset ylipäätään tapahtuvat. Ja onko ohjelma yhä puhdas ja
viitteellisesti läpinäkyvä?


\section{Siirrännän toteutus}

\subsection{IO-monadi}

Siirräntää varten Haskelliin on lisätty \code{IO}-tyyppi, joka joka ilmaisee toimintoa. \code{IO} on
yhdistelmätyyppi (engl. \emph{algebraic data type}), joka ottaa parametrikseen toisen tyypin, joka
ilmaisee toiminnon palautteen tyypin. Esimerkiksi funktio tyyppiä \code{IO Int} voi suorittaa
siirräntää ja palauttaa sen jälkeen kokonaislukuarvon.

\code{IO}-tyypille on myös määritelty \emph{Monad}-tyyppiluokan ilmentymä, joka tarkoittaa sitä,
että \code{IO}-tyypille on määritelty monadiset operaatiot \emph{bind}, \emph{then} ja
\emph{return}. Koska siirrännän toimiminen vaatii monadin ominaisuudet ja ne ovat \emph{IO}:n
leimallinen piirre, on tyypillistä puhua \emph{IO-monadista} IO-tyypin sijaan.

Samoin kuin monet aiemmat funktionaaliset ohjelmointikielet, myös Haskell käyttää
Hindley--Milner-tyyppijärjestelmää. Haskellissa on myös alusta alkaen ollut tyyppiluokat osana
kielen ilmaisuvoimaa. Kun IO-monadi päätettiin lisättiin Haskelliin 90-luvun alussa, oli se
teknisessä mielessä verrattain helppo operaatio, sillä monadit olivat luonteva toteuttaa
tyyppiluokkien avulla ja IO-tyyppi soveltui Hindley--Milner-järjestelmään sellaisenaan.

\subsection{Toiminto ja suoritus}

% We need a way to reconcile being with doing: an expression in a functional
% language denotes a value, while an I/O command should perform an action.

Funktionaalisessa kielessä lauseke ilmaisee arvoa, kun taas siirräntä on luonteeltaan suorite, joka
tulisi toteuttaa. Osana siirrännän ratkaisua on Haskellissa erotettu olemassa oleminen ja tekeminen
toisistaan. Tätä jakoa edustaa IO-monadi, joka on toiminto (engl. \emph{action}) ja jonka suoritus
(engl. \emph{perform}) voi toteuttaa siirräntää ja palauttaa arvon. Jatkossa \emph{toiminto}
tarkoittaa aina jotain funktiota, joka on \code{IO}-tyyppiä.

Erottelemalla toiminnon suorittamisesta \code{IO}-tyyppiset funktiot kapseloivat siirrännän siten,
että ne ovat yhteensopivat Haskellin puhtaan funktionaalisen paradigman kanssa säilyttäen
viitteellisen läpinäkyvyyden. Toisin sanoen toiminnot ovat tavallisia funktiota, joita voidaan
välittää argumenttina, koostaa osaksi korkeamman asteen funktioita (engl. \emph{function
composition}) ja yleisesti ottaen käsitellä kaikilla funktionaalisen kielen keinoilla.

\subsection{Siirrännän esimerkkiohjelma}

\begin{lstlisting}[float,label={lst:ioprog},caption={IO-ohjelma ja sen evaluaatio}]
main :: IO ()
main = getChar >>= putChar

getChar :: IO Char
putChar :: Char -> IO ()
(>>=)   :: IO a -> (a -> IO b) -> IO b

1. (>>=) getChar putChar
2. (IO a -> (a -> IO b) -> IO b) getChar putChar
3. (IO Char -> (Char -> IO b) -> IO b) (IO Char) putChar
4. ((Char -> IO b) -> IO b) putChar
5. ((Char -> IO ()) -> IO ()) (Char -> IO ())
6. IO ()
\end{lstlisting}

Listauksessa~\ref{lst:ioprog} on kolme osaa. Kahdella ensimmäisellä rivillä on esimerkkiohjelma,
joka lukee merkin vakiosyötteestä ja tulostaa sen vakiotulosteeseen. Ohjelma käyttää toimintoja
\code{getChar}, \code{putChar} ja $(>>=)$, joiden tyypit ovat kirjoitettu näkyviin. \code{getChar}
on toiminto, joka palauttaa \code{Char}-tyyppisen arvon. Vastaavasti \code{putChar} on toiminto,
joka ottaa \code{Char} arvon parametrinaan ja palauttaa tyhjän IO-toiminnon, jonka tyyppi on
\code{IO ()}. Haskellissa tyhjät sulkeet merkitsevät olematonta arvoa ja niiden käyttö tässä
yhteydessä on välttämätöntä, sillä IO-tyyppi vaatii tyyppiparametrin.

Listauksessa käydään myös esimerkkiohjelman evaluaatio vaihe vaiheelta läpi ohjelman käyttämien
tyyppien avulla. Evaluaation ensimmäisellä rivillä ohjelma on muutoin muuttumaton, mutta $(>>=)$
funktio on siirretty sisämerkinnästä parametriensa eteen, jotta esimerkkiä olisi helpompi seurata
myöhemmin.

Toisella rivillä $(>>=)$ on korvattu tyyppimääritelmällään. Tässä vaiheessa IO-tyyppi on vielä
polymorfisesssa muodossa, sillä lopullista tyyppiä ei vielä tiedetä. Toisin sanoen merkinnässä
käytetään yhä \code{a} ja \code{b} -tyyppiparametreja.

Kolmannella rivillä \code{getChar} on korvattu tyypillään ja samalla huomaamme että sen tyyppi sopii
sidosfunktion ensimmäisen parametrin tyyppiin, jos tyyppiparametri \code{a} saa arvon \code{Char}.
Tarkennamme tyyppimääritelmiä korvaamalla \code{IO a} -merkinnän tyypillä \code{IO Char}.

Neljännellä rivillä sievennämme lauseketta soveltamalla \code{getChar}-funktion ja poistamme sen
merkinnöistä. \fixme{Tässä vaiheessa Haskell-ohjelman suoritusta tapahtuu toiminnan suoritus.}

Viidennellä rivillä \code{putChar}-funktio korvataan tyypillään ja voimme jälleen tehdä
tyyppipäättelyn samalla tavalla kuin kolmannella rivillä. Huomaamme että tyyppiparametrin \code{b}
tyypiksi sopii \code{()} ja se mahdollistaa \code{putChar}-funktion soveltamisen. Tässä vaiheessa
alkuperäisestä sidosfunktiosta on jäljellä enää toinen parametri ja paluuarvo. Kuudennella rivillä
funktio \code{putChar} on sovellettu sidosfunktioon ja jäljelle jää enää paluuarvon tyyppi \code{IO
()}.

Siirrännän lisäksi tämä esimerkki kattaa monta yleistä funktionaalisen kielen ominaisuutta, jotka
ovat syytä mainita selkeyden vuoksi. Ohjelman sieventäminen tyyppien avulla on vastaavanlainen
operaatio kuin kääntäjän tekemä tyyppipäättely Haskell-ohjelmaa käännettäessä. Esimerkin
sieventyminen tyypiksi \code{IO ()} ei ole sattuma, sillä se on vaatimus kaikille
Haskell-ohjelmille, sillä kaikkien omavaraisten Haskell-ohjelmien suoritus on sidottu
\code{main}-funktioon, jonka tyyppi on \code{IO ()}.

Toinen merkittävä funktionaaliseen kieleen liittyvä havainto on se kuinka ohjelmaa suoritetaan.
Imperatiivisessa kielissä suoritus yleensä koostu toisistaan riippumattomien lauseiden ajamisesta
siten, että suorituksen järjestys määräytyy ensisijaisesti lauseiden järjestyksestä lähdekoodissa.
Sen sijaan funktionaalisen kielen ohjelma on itsessään kokonainen funktio, joka muodostuu
pienemmistä funktioista. Ohjelman suoritus on sieventämistä, joka tapahtuu funktioita soveltamalla
ja suoritusjärjestys ohjautuu kielen sievennyssääntöjen mukaisesti.

\subsection{Toimintojen yhdistäminen}

Siirräntä ei ole triviaalia puhtaassa ja laiskassa ohjelmointikielessä, sillä kieli itsessään ei
määrittele missä järjestyksessä siirräntä tulee suorittaa eikä edes sitä, suoritetaanko sitä
ollenkaan. Itse asiassa päin vastoin Haskellin määritelmä sanoo, että kieli on löyhä
evaluaatiojärjestyksestään (engl. \emph{non-strict evaluation semantics}). Monadit, ja tässä
tapauksessa erityisesti IO-monadi, ratkaisevat toimintojen suoritusjärjestyksen ongelman siten, että
monadiset operaatiot liittävät toiminnot yksiselitteisesti osaksi evaluaatiojärjestystä.

Edellisessä esimerkkiohjelmassa \code{getChar}-funktio suoritettiin ennen \code{putChar}-funktiota,
sillä \code{putChar} vaati argumentikseen \code{getChar}-funktion paluuarvon. Suoritus ei olisi
voinut tapahtua käänteisessä järjestyksessä, sillä näiden kahden toiminnon välillä on
tietoriippuvuus, joka luotiin $(>>=)$-metodin avulla. Evaluaatiojärjestyksen takaaminen on yksi
IO-monadin tärkeistä ominaisuuksista.

Toinen merkittävä IO-monadin tehtävä on, että se tarjoaa viitekehyksen, joka tekee siirrännän
yhdistämisestä Haskell-ohjelmiin verrattain helppoa ja ymmärrettävää. Vaikka siirrännän
esimerkkiohjelma oli hyvin lyhyt, sen perusteellinen läpikäynti paljasti monisärmäisen rakennelman,
jonka ymmärtäminen edellyttää kattavan teoreettisen tietopohjan. Siitä huolimatta on
perusteltavissa, että ilmaisu

$$ getChar >>= putChar $$

on sekä yksinkertainen että intuitiivinen merkityksen huomioon ottaen. Toisin sanoen, vaikka
Haskellin juuret ovat syvällä kategoriateoriassa, ohjelmoijan harteille ei ole sysätty kognitiivista
kuormaa muistaa IO-monadin yksityiskohtia päivittäisessä ohjelmointityössään.

\subsection{Imperatiivinen tyyli}
% Viitteeksi ehkä jotain Wadleria (kuten Comprehending Monads)

% The essence of monad is thus separation of composition timeline
%   from the composed computation's execution timeline,
%   as well as the ability of computation to implicitly carry extra data,
%   as pertaining to the computation itself, in addition to its one output,
%   that it will produce when run.

Imperatiivinen ohjelmointimalli soveltuu hyvin siirräntään, sillä imperatiivisen koodin rakenne ja
ohjelman semantiikka vastaavat toisiaan. Haskelliin on siksi lisätty \verb|do|-merkintätapa (engl.
\emph{do-notation}), joka mahdollistaa Haskell-ohjelmakoodin kirjoittamisen imperatiiviseen tyyliin,
mutta hyödyntäen monadeja ja funktionaalista semantiikkaa. Do-merkintätapa on syntaktista sokeria,
joten se ei lisää eikä vähennä Haskellin kielellistä ilmaisuvoimaa.

Do-merkintätapa lisää Haskelliin avainsanat \code{do} ja \verb|<-|. Avainsana \code{do} aloittaa
koodilohkon, joka rajataan ympäröivästä ohjelmasta sisentämällä, ja jonka kontekstissa kääntäjä
pelkistää (engl. \emph{desugar}) rivinvaihdot monadisiksi yhdistefunktioiksi. Samalla tavalla kuin
usein imperatiivisissa ohjelmissa rivi merkitsee yhtä itsenäistä ilmaisua, \code{do}-lohkossa rivi
merkitsee yhtä toimintoa.

Avainsana \verb|<-| luetaan \emph{bind} tai \emph{sidos}. Nimi on sama kuin funktiolla \verb|(>>=)|,
sillä molemmat suorittavat samaa monadista tehtävää eli sitovat toiminnosta tulleen paluuarvon. Ne
eivät kuitenkaan ole identtiset, sillä siinä missä \verb|(>>=)| välittää paluuarvon seuraavalle
toiminnolle, \verb|<-| ottaa paluuarvon ja antaa sille nimen, jota voidaan käyttää laskennassa
myöhemmin.

\begin{lstlisting}[float,label={lst:desugared},caption={Funktionaalinen tyyli}]
main :: IO ()
main = putStrLn "Mikä on nimesi?"
       >>
       getLine
       >>=
       \nimi -> putStrLn ("Hei " ++ nimi ++ "!")
\end{lstlisting}

\begin{lstlisting}[float,label={lst:donotation},caption={Imperatiivinen tyyli ja do-merkintätapa}]
main :: IO ()
main = do
  putStrLn "Mikä on nimesi?"
  nimi <- getLine
  putStrLn ("Hei " ++ nimi ++ "!")
\end{lstlisting}

\emph{Do}-notaatio lienee helpoin hahmottaa esimerkin avulla. Listauksessa~\ref{lst:desugared} on
esimerkkiohjelma, joka kysyy käyttäjän nimeä, lukee syötteen ja tulostaa tervehdyksen. Koska alussa
tapahtuva tulostus ei palauta merkityksellistä arvoa, käytetään funktiota \code{then} yhdistämään
toimintoja. Sen sijaan \code{getLine} palauttaa merkkijonon, joten sen paluuarvo passataan eteenpäin
funktion \code{bind} avulla. Esimerkin viimeisellä rivillä on nimetön funktio, jota kutsutaan myös
lambdaksi. Se ottaa parametrikseen \code{getLine}-funktiolta tulleen merkkijonon, antaa sille nimen
\code{nimi} ja tulostaa tekstiä käyttäen saamaansa parametria.

Vaikka sama ohjelma on mahdollista kirjoittaa käyttämättä lambda-ilmaisua, on sen avulla verrattain
helppo antaa nimi funktion \code{getLine} paluuarvolle. Vastaavasti paluuarvon nimeäminen ei ole
välttämätöntä, mutta ohjelmoitaessa tulee jatkuvasti vastaan tilanteita, joissa asioiden nimeäminen
on hyödyllistä, jotta niihin on helppo myöhemmin viitata tai jotta koodi pysyy helpommin luettavana
ja ymmärrettävänä.

\fixme{Pitäisikö $<-$ nimetä sidosnuoleksi ja käyttää sitä?}

Listauksessa~\ref{lst:donotation} on sama ohjelma, mutta tällä kertaa se on kirjoitettu
\code{do}-merkintätapaa käyttäen. Do-merkintätavan takia monadisia sidosfunktioita ei tarvitse
kirjoittaa täsmällisesti näkyviin, sillä kääntäjä päättelee niiden tarpeen rivinvaihtojen ja
\verb|<-|-sidoksen perusteella. Toiminnon \code{getLine} palauttama arvo sidotaan \verb|<-|:lla
nimeen syntaksilla, joka on samankaltainen verrattuna minkä tahansa imperatiivisen ohjelmointikielen
tapaan tallentaa arvo johonkin muuttujan.

% Huomattavaa kuitenkin on, ettei Haskellin yhteydessä puhuta muuttujista, sillä
% \code{nimi} ei ole muuttuja, sillä sen arvoa ei voi myöhemmin muuttaa.

Ilman \emph{do}-merkintää paluuarvon nimeäminen vaati selkeästi monimutkaisemman ilmaisun ja vaati
ylimääräisen anonyymin funktion käyttöä. Sen sijaan jäljemmässä esimerkissä toiminto ja sen
paluuarvon nimeäminen tapahtuvat luontevasti samalla rivillä. Yleisesti ottaen \emph{do}-merkintä
tarjoaa syntaksin, joka on intuitiivinen ja ennestään tuttu lähes mitä tahansa muita
ohjelmointikieliä käyttäneille, mutta jonka avulla on helppo hyödyntää Haskellin ja funktionaalisen
ohjelmoinnin poikkeuksellisia ominaisuuksia: \fixme{Ei kaksoispistettä, mutta kuinka?}
Tyyppipäättely ja -turvallisuus auttavat ohjelmointivirheiden havaitsemista. Funktionaalisuuteen
kuuluva funktioiden ja toimintojen yhdistäminen toimii. Laiskuus ja puhtaus säilyvät
käyttökelpoisina.

% FIXME Monads can be thought of as composable computation descriptions.
%       Vielä ei vissiin ole kerrottu tätä "konteksti" asiaa

% i/o juttuja:
%  kaikki sitoutuu lopulta pääohjelmaan main (ehkä oma subsection)
%  IO ei paljasta konstruktoria, joten turvallinen (ehkä oma subsection)
%   -> https://stackoverflow.com/a/19093720
%  siirräntä tehdään tarvittaessa ja järjestetysti

\subsection{Oikea maailma}

Aiemmin on mainittu IO-monadin luoma viitekehys ja se, että monadiset operaatiot -- erityisesti
\code{(>>=)} -- ratkaisevat laiskuuteen ja laskennan järjestykseen liittyvät haasteet. Mutta mikä
mekanismi tarkalleen ottaen määrää evaluaatiojärjestyksen? Osa vastausta on ollut omaksua ajatus
Haskell-ohjelman ulkopuolisesta maailmasta osaksi Haskellin semantiikkaa. Toisin sanoen
Haskell-kääntäjä tuntee konseptin oikeasta maailmasta. Oikeaa maailmaa kutsutaan Haskellin
kontekstissa joskus nimellä \emph{RealWorld}, mutta jatkossa käytetään typistettyä nimitystä
\emph{World}.

\begin{lstlisting}[float,label={lst:world},caption={IFP, IO-tyyppi}]
type IO a    = World -> IORes a
data IORes a = MkIORes a World
\end{lstlisting}

Ohjelman ulkopuolisen maailman ottaminen huomioon ohjelmointikielen tasolla voi vaikuttaa ylettömän
korkealentoiselta, mutta tosiasiassa se ratkaisee puhtauteen liittyvät ongelmat.
Listauksessa~\ref{lst:world} on ote julkaisusta Imperative Functional Programming\cite{ifp93}, johon
Haskellin IO-monadin toteutus pohjautuu. Määrittelyssä käytetyn tyyppialiaksen sovelmisen jälkeen
IO-tyypin määritelmä saadan yksinkertaistettua muotoon:

$$ IO\; a = World \arr (a,\; World) $$

Luettuna IO on siis yhdistelmätyyppi ja funktio, jonka parametri on maailma ja arvo on pari, joka
koostuu toiminnon paluuarvosta ja maailmasta. Käsitteellisesti siis Haskell-ohjelman toiminto ottaa
syötteekseen koko tunnetun maailman ja palauttaa uuden maailman, jonka tilaa on muutettu kyseisen
toiminnon verran. Ajatus voi vaikuttaa liioiteltulle, mutta se on yhteensopiva viitteellisen
läpinäkyvyyden kanssa, joka vastaavasti on kriittinen vaatimus puhtauden säilyttämiseksi.

Tietysti koko tunnetun maailman muunnos on laskennallisesti liian mahtipontinen suorite, joten
\emph{World} on Haskell-kääntäjälle abstrakti tietotyyppi. Tämä tarkoittaa sitä, että toisaalta
\emph{World} on kääntäjälle todellinen, sillä sitä käytetään lähdekoodissa IO-monadia
määriteltäessä, mutta toisaalta \emph{World} ei koskaan päädy käännettyyn Haskell-ohjelmaan eikä se
näy Haskell-ohjelmassa ajon aikana, sillä se optimoituu pois käännösvaiheessa.

\begin{lstlisting}[float,label={lst:bindio},caption={IFP, bindIO:n määritelmä}]
bindIO m k w = case (m w) of
  MkIORes a w' -> k a w'
\end{lstlisting}

Ennen kuin kääntäjä poistaa kyseisen abstraktion on sillä konkreettinen tehtävä toimintojen
ketjuttamisessa. Listauksessa~\ref{lst:bindio} on \code{bindIO}-funktion määritelmä~\cite{ifp93},
joka on käytännössä toteutus sidosfunktiolle \code{(>>=)}. \code{bindIO}:n argumentit ovat monadiset
toiminnot (\code{m} ja \code{k}) sekä \emph{World} (\code{w}). Ensimmäinen toiminto suoritetaan
yhdessä maailman tilan kanssa ja tulos sovitetaan IO-paluuarvoksi \code{a} ja uudeksi maailman
tilaksi \code{w'}. Sovituksen onnistuessa uusi maailman tila palautetaan muiden paluuarvojen mukana.

Kun abstraktia maailman tilaa kuljeteteaan siirräntätoimintojen mukana läpi Haskell-ohjelman,
saadaan aikaan suoraviivainen linkitetty toimintojen ketju. Tai tarkemmin ilmaistuna \emph{World}
luo tietoriippuvuuden (engl. \emph{data dependency}) kaikkien ohjelman toimintojen välille jokaisen
toiminnon sekä kuluttaessa että palauttaessa maailman tilan. Koska ketjun aiemman toiminnon
paluuarvoa tarvitaan ketjun seuraavan toiminnon toteutukseen, on Haskell-ohjelman välttämätöntä
evaluoida toiminnot järjestyksessä.

Tässä vaiheessa voimme palauttaa mieleen listauksessa~\ref{lst:ratkaisu} olleen esimerkkiohjelman.
Siinä toimintojen järjestyksen ongelma ratkaistiin luomalla ylimääräiset kokonaislukuparametrit,
joita välittämällä suoritus pakotettiin toivottuun järjestykseen. IO-monadi ja sen kontekstissa
käytetty abstrakti tietotyyppi \emph{World} ovat loogisesti sama ratkaisu laiskuuden
tarjoamaan haasteeseen. Oleellinen ero on, että monadi kätkee ratkaisun toteuksen ohjelman syntaksin
tasolla ja kääntäjä optimoi sen pois haittaamasta ohjelmen suoritusta.

\subsection{Pääohjelma} %% IFP.2

Monadien tarjoama viitekehys auttaa ratkaisemaan laiskuuteen ja laskennan järjestykseen liittyvät
haasteet, mutta siirräntä on silti konseptuaalisesti vaikea asia Haskellille. Siirräntä on ennen
kaikkea sivuvaikutusten aiheuttamista, mutta Haskell on puhdas ja puhtaus tarkoittaa sivuvaikutusten
välttämistä. Ristiriita on ilmeinen. Mikä siis pakottaa Haskell-ohjelman toteuttamaan siirräntää
ylipäätään?

% Toiminnon ja suorituksen erottelu johtaa luonnollisesti kysymykseen siitä, mikä saa aikaan
% toimintojen suorittamisen?

Haskellin ratkaisu on määritellä koko ohjelma toiminnoksi nimeltään \code{main}, jota voidaan kutsua
pääohjelmaksi ja joka on tyyppiä \code{IO ()}. Toiminnon \code{main} suoritus vastaa koko ohjelman
suorittamista. Haskell-ohjelma on funktionaalisesta semantiikasta johtuen mielivaltainen joukko
matalamman tason toimintoja, jotka on ketjutettu monadisilla operaatioilla pääohjelman tyypin takia.
Toisin sanoen jotta koko ohjelman voi suorittaa, on seurattava toimintojen riippuvuusketjua ja
evaluoitava ketjun kaikki toiminnot järjestyksessä.

Pääohjelman käsite on yleinen myös imperatiivisissa ohjelmointikielissä, joissa sitä käytetään
suorituksen lähtöpisteenä. Haskellin \code{main} vastaa intuitiivisesti imperatiivisten kielien
pääohjelmaa, mutta semanttisesti niillä ei ole yhtäläisyyksiä. Voidaan ajatella että
funktionaalisuuden takia pääohjelma ei niinkään ole suorituksen alkupiste vaan pikemminkin lopetus,
sillä Haskell pyrkii evaluoimaan koko ohjelman, jotta lopulta saavutetaan \code{main}-toiminnon
paluuarvo.


% motivaatio / miksi
% miksi IO: turvallinen, puhtaus
% lopulta IO on ccall, mutta monadit tuovat sivuvaikutukset funktionaaliseen kieleen
% IO ei paljasta konstruktoria, joten turvallinen (ehkä oma subsection)

\section{Yhteenveto}

\fixme{TODO}

%  Kertausta
% Kaikki siirräntä Haskellissa on IO-tyyppiä, joten kaikki siirräntä tapahtuu IO-monadin kontekstissa.
% Kontekstilla tarkoitetaan sitä että monadiset operaatiot yhdistävät jne. /World/ on IO-tyypin
% rakennelma, joten se on siellä aina läsnä.
% Monadeilla on saavutettu paljon muutakin kuin siirräntä

% Yleensä hieman johdantoa lyhyempi.
% Muistuttaa mieleen tutkimuskysymyksen, mainitsee tärkeimmät tulokset ja niiden perusteet.
%  Keskittyy impaktiin ja esimerkiksi suosituksiin. Ei vain summeeraa luku kerrallaan aikaisempaa
%  tekstiä.


%  %%  %  %%  %  %%  %  %%  %  %%  %  %%  %  %%  %  %%  %  %%  %

% Jones: Tackling the Awkward Squad
% doneIO & seqIO :: monoid  ||  bindIO & unitIO :: monad

% Monadit ovat semanttinen asia. Haskell ottaa kategoriateorian monadit ja luovat Monadin. Toisin
% sanoen Haskell mahdollistaa ohjelmoijan määritellä oman laskennallisen ympäristönsä. Eli:
% "[Moggi(1989)] offers the interpretation of a programming language into one monad, but Haskell
% reifies monads with Monad. Reification is taking a concept used to describe the semantics of a
% language and inserting it into the language itself. Monad lets Haskell programmers define their
% own computational universes."

% 4.3 No way out [https://wiki.haskell.org/All_About_Monads]
% The wonderful feature of a one-way monad is that it can support side-effects in its monadic
% operations but prevent them from destroying the functional properties of the non-monadic portions
% of the program.

\bibliographystyle{babalpha-lf}
\bibliography{viitteet}

\end{document}
